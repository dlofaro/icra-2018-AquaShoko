
\begin{figure}[h]
\centering
\includegraphics[width=1.0\columnwidth]{./img/aquaPod-test-one-start.JPG}
\caption{Aquapod version 3.1 moments after submersion}
\label{fig:pod in bucket}
\end{figure}

\begin{figure}[h]
\centering
\includegraphics[width=1.0\columnwidth]{./img/aquaPod-test-one-rust.JPG}
\caption{Aquapod version 3.1 after first water resistance test}
\label{fig:pod rust}
\end{figure}


\subsection{Aquapod water resistance testing}


\subsubsection{Shallow Submersion Test}
Test water resistance of resistance of Aquapod in shallow water.
Aquapod V3.1 without an actuator installed was left in a 5 gallon bucket filled with tap water from 23/6/2017 12:30PM to 25/6/2017 12:30PM. 
Test was successful. No evidence of water ingress into the Aquapod or electrical connectors. 

Bucket Test observations
\begin{itemize}
    \item Rust on the surface of black oxide steel hardware
    
    \item Rust on the surface of steel bearing. 

    \item The bearing spun rough after test

\end{itemize}

\begin{figure}[h]
\centering
\includegraphics[width=1.0\columnwidth]{./img/aquaPod-test-two-pressureCheck.JPG}
\caption{Chamber internal gauge pressure during low pressure chamber test}
\label{fig:test two pressure check}
\end{figure}


\subsubsection{Low Pressure Chamber Test}
Aquapods are expected to reliably function at a depth of 3 meters. In order to simulate the depth a pressure chamber was was used. Using equation~\ref{eq:head} the gauge pressure at 3m head to be 29.4kPa(4.2psi)

\begin{equation}\label{eq:head}
    P = \rho g h \\
\end{equation}
\begin{align*}
    P & = \text{Gauge Pressure}  \\
    \rho & = 1000 kg/m^3 \\
    g & = 9.81 m/s^2 \\
    h & = depth  \\
\end{align*}

A factor of safety of 2.5 or higher is desired for water resistance. A minimum pressure benchmark for a successful test was calculated to be 73.6kPa(10.7psi) which simulates a depth of 7.5m.
The test took place on 27/6/2017 from 11:30am to 4:00pm which are the respective times of pressurization and depressurization.
The pressure chamber filled with tap water and Aquapod V3.1 without an actuator installed was submerged inside.
The source air pressure was initially adjusted to 241kPa(35psi) gauge and used initial pressurization of the chamber.
The source air was disconnected from the chamber and a gauge was used to measure the pressure inside the chamber to be 103kPa(15psi). 
The gauge was left connected for 5 minutes the pressure was observed to remained stable. 
The source air was then adjusted to ~20psi gauge and left attached to the pressure chamber for the remainder of the test. 
Wet spots where observed indicating minor leakage was around the top of the pressure chamber.
No evidence of moisture inside the Aquapod or any of the electrical connectors. 
Test proved the Aquapod is water resistant beyond the to a 7.5m depth which satisfies the desired factor of safety of 2.5.
This test also suggests Aquapods maybe water resistant to or beyond 24m head, however due to the air leakage this claim cannot be confirmed and additional future test are required.

Low pressure chamber test observations
\begin{itemize}
    \item High pressure capable chamber is required for future tests
    
    \item Rust on the surface of steel bearing 
    
    \item More tests are required to determine the water resistance limit Aquapods

    \item Design needs to be slightly modified for ease of dis-assembly

    \item Due to difficulty during dis-assembly the mating surfaces of Pod-Bottom and Pod-Top were damaged
    
    \item An Aquapod must be tested in accord to IP68 standard procedures 
    
\end{itemize}

