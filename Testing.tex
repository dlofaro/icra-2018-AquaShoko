

\subsection{Aquapod water resistance testing}


\subsubsection{Shallow Submersion Test}
Test the water resistance of Aquapod in a shallow body of water.
Aquapod V3.1, without an actuator installed, was left in a 5 gallon bucket filled with tap water for a duration of 48 hours. 
No evidence of water ingress into the Aquapod nor electrical connectors; concluding a successful shallow submersion test.

Shallow Submersion Test observations:
\begin{itemize}
    \item Rust on the surface of black oxide steel hardware
    
    \item Rust on the surface of steel bearing

    \item The bearing spun rough after test

\end{itemize}

\begin{figure}[h]
\centering
\includegraphics[width=1.0\columnwidth]{img/rust.png}
\caption{(LEFT)Aquapod version 3.1 after first water resistance test, note the presence of rust.  (RIGHT)Aquapod version 3.1 moments after submersion, note the absence of rust.}
\label{fig:pod in bucket}\label{fig:pod rust}
\end{figure}





\subsubsection{Low Pressure Chamber Test}
Aquapods are expected to function reliably at a depth of 3m. A pressure chamber was used in order to simulate the proper depth. Using equation~\ref{eq:head}, the gauge pressure at 3m head to be 29.4kPa (4.2psi).

\begin{equation}\label{eq:head}
    P = \rho g h \\
\end{equation}

\noindent where P is gague pressure, $\rho$ is $1000 kg/m^3$, g is gravity, and h is the depth.

A factor of safety of 2.5 or higher is desired for water resistance. A minimum pressure benchmark for a successful test was calculated to be 73.6kPa (10.7psi), which simulates a depth of 7.5m.
The test took place over a 5 hour time span.
The pressure chamber was filled with tap water, and then the Aquapod V3.1, without an actuator installed, was submerged inside.
The source air pressure was initially adjusted to 241kPa (35psi) gauge and used initial pressurization of the chamber.
The source air was disconnected from the chamber and a gauge was used to measure the pressure inside the chamber to be 103kPa (15psi). 
The gauge was left connected for 5 minutes the pressure was observed to remained stable. 
The source air was then adjusted to 138kPa (20psi) gauge and left attached to the pressure chamber for the remainder of the test. 
Wet spots were observed, which indicated minor leakage occurred around the top of the pressure chamber; however, no evidence of moisture was found inside the Aquapod or any of the electrical connectors.
The test proved the Aquapod is water resistant beyond the 7.5m depth, which satisfies the desired factor of safety of 2.5.
This test also suggests Aquapods may be water resistant to or beyond 24m head; however, due to the air leakage, this claim cannot be confirmed and additional future testing is required.

Low pressure chamber test observations:
\begin{itemize}
    \item High pressure chamber is required for future tests
    
    \item Rust on the surface of steel bearing 
    
    \item More tests are required to determine the water resistance limit

    \item Design needs to be slightly modified for ease of dis-assembly

    \item Due to difficulty during dis-assembly, the mating surfaces of Pod-Bottom and Pod-Top were damaged
    
    \item An Aquapod must be tested in accord to IP68 standard procedures 
    
\end{itemize}

\begin{figure}[h]
\centering
\includegraphics[width=0.5\columnwidth]{./img/aquaPod-test-two-pressureCheck.JPG}\includegraphics[width=0.5\columnwidth]{./img/qr.jpg}
\caption{(RIGHT) Chamber internal gauge pressure during low pressure chamber test (LEFT) Video of AquaShoko being tested running the controller in \ref{sec:stable} under 1.5 $m$ of water.}
\label{fig:test two pressure check-1}\label{fig:video}
\end{figure}

\subsubsection{Physical Robot Underwater Test}
The physical robot was then tested for its ability to operate underwarter.  
Figure~\ref{fig:underwarter} shows the test of the robot running the body orientation algorithm defined in \ref{sec:stable}.
The depth of the robot is approximately 1.5 $m$.
The robot was tested for 1 hours continuously underwarter.
Figure~\ref{fig:video} is a link to the video for the AquaShoko test depicted in Figure~\ref{fig:underwarter} on the real robot under 1.5 $m$ of water.







