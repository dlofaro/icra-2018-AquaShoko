Submersibles operating near the oceans floor use cameras as one of their primary sensors.
This is especially true when operating when a low signature is required, i.e. not being able to use active sonar.
The submersible's cameras become obstructed by silt and other debris when the ground is disturbed by the thrust from the submerged vehicles screws/thrusters.
In may cases these visual sensors become useless in these conditions.
The overall goal of the overarching project is to create a system that can operate underwater in low signature environments while keeping visibility, mobility, and dexterity. 

This part of the project, and this paper, focuses on the initial design of a fully waterproof robot that can operate near/on the sea floor.
The robot is a multi-legged, where each leg is long and thin to reduce disturbances and vortexes caused by the robot's motion underwater.
In the future (but not in this paper) the legs will double as manipulators.
Within the design is the ability to land on its feet by being able to move its center of buoyancy.
This is so the robot can be deployed by dropping if off of the side of a surface vessel.
Control methods for achieving this behavior is included in this document.
All design choices, FEA, load analysis, and other pertinent specification requirements are also included.
All parts have been tested and simulated depths beyond the desired ratings.
The result of this work is the design, fabrication, and creation of the IP-68@3$m$ rated quadruped that has a tip to tip leg span of more than 2$m$.
The robot that was created can be seen in Figure~\ref{fig:cover}.
The final robot has been tested and fully functions underwater and can be see in Figure~\ref{fig:underwarter}.



\begin{figure}[!t]
\centering
\includegraphics[width=1.0\columnwidth]{img/aquashoko-2.pdf}
\includegraphics[width=1.0\columnwidth]{img/aquaShoko-fea-cad.pdf}
\caption{AquaShoko Robot: An IP-68 at 3$m$ rated underwater 12 degree of freedom quadruped designed for near/on sea floor for low signature underwater exploration.  The robot's long thin legs are designed to reduce disturbances and vortexes caused by the robot's motions. Future goals include adding manipulation abilities and autonomous operations. (Top Left) Real Robot, (Top Right) Simulated Robot in Gazebo Sim, (Bottom Left) CAD Model, and (Bottom Left) Finite Element Analyses of robot model. }
\label{fig:cover}
\end{figure}  
